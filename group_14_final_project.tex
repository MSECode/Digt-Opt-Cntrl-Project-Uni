\documentclass[11pt]{article} 
% ~~~~~~~~~~~~~~~~~~~~~~~~~~~~~~~~~~~~~~~~~~~~~~~~~~~~~ %
\input{./Scripts/packages}								
\input{./Scripts/ridefinitions}							
\input{./Scripts/figuresgraphicalsettings}				
\input{./Scripts/tablesgraphicalsettings}				
\input{./Scripts/newcommands}							
% ~~~~~~~~~~~~~~~~~~~~~~~~~~~~~~~~~~~~~~~~~~~~~~~~~~~~~ %

\title{\Huge ELEC-E8101 Group project: \\ Lab A report \\ Group 14}
\date{\today}
\author{LOSI Jacopo, MAJOR Gabor, SALJOUGHI Nicola }


\begin{document}
\maketitle

\begin{instructions}
For this lab report there is no size limit on your report, but try to be concise.	
\end{instructions}



\subsection*{Reporting of Task 4.1}

We want to express the linearized EOM in the form of a general linear time-invariant (LTI) SS system:
\begin{equation*}
\begin{cases}
	\dot{\mathbf{x}} = A\mathbf{x} + Bu\\
	y = C\mathbf{x} + Du
\end{cases}
\end{equation*}
The input is the voltage applied to the motors and the measurement is the angular deviation of the balancing robot from the vertical upright position:
\begin{align*}
	u &= v_m \\
	y &= \theta_b
\end{align*}
We choose state $\mathbf{x}$ as:
\[
\begin{bmatrix}
	x_w\\
	\dot x_w\\
	\theta_b\\
	\dot \theta_b
\end{bmatrix}
\]
The obtained $A$, $B$, $C$ and $D$ matrices are here presented in parametric form:
$$A =$$

\newcommand\scalemath[2]{\scalebox{#1}{\mbox{\ensuremath{\displaystyle #2}}}}


\[
\left[ \scalemath{0.55}{ \begin {array}{cccc} 0&1&0&0\\ \noalign{\medskip}0&-2\,{\frac 
{ \left( l_{b}\, \left( l_{b}+l_{w} \right) m_{b}+I_{b} \right) 
 \left( K_{e}\,K_{t}+1/2\,b_{f}\,R_{m} \right) }{ \left(  \left( I_{b}
\,{l_{w}}^{2}+{l_{b}}^{2} \left( {l_{w}}^{2}m_{w}+I_{w} \right) 
 \right) m_{b}+I_{b}\, \left( {l_{w}}^{2}m_{w}+I_{w} \right)  \right) 
R_{m}}}&-{\frac {{m_{b}}^{2}{l_{b}}^{2}{l_{w}}^{2}g}{ \left(  \left( m
_{w}+m_{b} \right) {l_{w}}^{2}+I_{w} \right) I_{b}+{l_{b}}^{2}m_{b}\,
 \left( {l_{w}}^{2}m_{w}+I_{w} \right) }}&2\,{\frac {l_{w}\, \left( m_
{b}\,{l_{b}}^{2}+m_{b}\,l_{b}\,l_{w}+I_{b} \right)  \left( K_{e}\,K_{t
}+1/2\,b_{f}\,R_{m} \right) }{ \left(  \left(  \left( {l_{b}}^{2}m_{w}
+I_{b} \right) m_{b}+I_{b}\,m_{w} \right) {l_{w}}^{2}+I_{w}\, \left( m
_{b}\,{l_{b}}^{2}+I_{b} \right)  \right) R_{m}}}\\ \noalign{\medskip}0
&0&0&1\\ \noalign{\medskip}0&2\,{\frac { \left(  \left( m_{w}+m_{b}
 \right) {l_{w}}^{2}+m_{b}\,l_{b}\,l_{w}+I_{w} \right)  \left( K_{e}\,
K_{t}+1/2\,b_{f}\,R_{m} \right) }{l_{w}\, \left(  \left(  \left( {l_{b
}}^{2}m_{w}+I_{b} \right) m_{b}+I_{b}\,m_{w} \right) {l_{w}}^{2}+I_{w}
\, \left( m_{b}\,{l_{b}}^{2}+I_{b} \right)  \right) R_{m}}}&{\frac {m_
{b}\, \left(  \left( m_{w}+m_{b} \right) {l_{w}}^{2}+I_{w} \right) gl_
{b}}{ \left(  \left( {l_{b}}^{2}m_{w}+I_{b} \right) m_{b}+I_{b}\,m_{w}
 \right) {l_{w}}^{2}+I_{w}\, \left( m_{b}\,{l_{b}}^{2}+I_{b} \right) }
}&-2\,{\frac { \left(  \left( m_{w}+m_{b} \right) {l_{w}}^{2}+m_{b}\,l
_{b}\,l_{w}+I_{w} \right)  \left( K_{e}\,K_{t}+1/2\,b_{f}\,R_{m}
 \right) }{ \left(  \left(  \left( {l_{b}}^{2}m_{w}+I_{b} \right) m_{b
}+I_{b}\,m_{w} \right) {l_{w}}^{2}+I_{w}\, \left( m_{b}\,{l_{b}}^{2}+I
_{b} \right)  \right) R_{m}}}\end {array}} \right]\]
\medskip
\medskip
\medskip
\medskip
\begin{equation*}
B = 
\left[ \scalemath{0.8}{\begin {array}{c} 0\\ \noalign{\medskip}2\,{\frac {l_{w}\,K_{t
}\, \left( m_{b}\,{l_{b}}^{2}+m_{b}\,l_{b}\,l_{w}+I_{b} \right) }{
 \left(  \left(  \left( m_{w}+m_{b} \right) {l_{w}}^{2}+I_{w} \right) 
I_{b}+{l_{b}}^{2}m_{b}\, \left( {l_{w}}^{2}m_{w}+I_{w} \right) 
 \right) R_{m}}}\\ \noalign{\medskip}0\\ \noalign{\medskip}-2\,{\frac 
{K_{t}\, \left(  \left( m_{w}+m_{b} \right) {l_{w}}^{2}+m_{b}\,l_{b}\,
l_{w}+I_{w} \right) }{ \left(  \left(  \left( {l_{b}}^{2}m_{w}+I_{b}
 \right) m_{b}+I_{b}\,m_{w} \right) {l_{w}}^{2}+I_{w}\, \left( {l_{b}}
^{2}m_{b}+I_{b} \right)  \right) R_{m}}}\end {array}} \right]
\end{equation*}
\medskip
\medskip
\medskip
\begin{equation*}
C = \left[ \scalemath{0.8}{\begin {array}{cccc} 0&0&1&0\end {array}} \right] 
\end{equation*}
\medskip
\medskip
\begin{equation*}
D = \left[ \scalemath{0.8}{\begin {array}{c} 0 \end {array}} \right] 
\end{equation*}
Here we report matrices $A$, $B$, $C$ and $D$ in numeric form:
\begin{equation*}
A =  \left[ \scalemath{0.8}{\begin {array}{cccc} 0&1&0&0\\ \noalign{\medskip}0&-
 773.7853734&- 6.573516819& 16.24949284\\ \noalign{\medskip}0&0&0&1
\\ \noalign{\medskip}0& 3313.238430& 63.07193800&- 69.57800702
\end {array}} \right] 
\end{equation*}
\begin{equation*}
B =  \left[ \scalemath{1}{\begin {array}{c} 0\\ \noalign{\medskip} 36.59795686
\\ \noalign{\medskip}0\\ \noalign{\medskip}- 156.7072230\end {array}}
 \right] 
\end{equation*}
\medskip
\medskip
\medskip
\begin{equation*}
C = \left[ \scalemath{0.8}{\begin {array}{cccc} 0&0&1&0\end {array}} \right] 
\end{equation*}
\medskip
\medskip
\begin{equation*}
D = \left[ \scalemath{0.8}{\begin {array}{c} 0 \end {array}} \right] 
\end{equation*}

\subsection*{Reporting of Task 4.2}
The transfer function of an LTI SS system is given by the formula:
\begin{equation*}
	G(s) = C(sI - A)^{-1}B + D
\end{equation*}
The result reported has been obtained by computing an analytical expression of the trasnfer function and then substituting the data. 
\begin{align*}
	G(s) = \scalemath{0.9}{0.001240873320\,{\frac {s}{ \left( s+ 843.4002 \right)  \left( s-
 5.6790 \right)  \left( s+ 5.6422 \right) }}
}
\end{align*}

At first we used the numeric forms of the matrices $A$, $B$, $C$ and $D$ to define the SS system in \texttt{MATLAB} and then to compute the transfer function. This however resulted in strange results, giving us additional poles and complex conjugate zeros (with real part $=0$ and imaginary part $=\pm i$). We then tried to go fully analytical and we got much more reasonable results; in this phase we also compared the results obtained using \texttt{Maple} and \texttt{MATLAB}, concluding that they are quite different since these softwares run in completely different manners. From this point onwards we only used \texttt{MATLAB}. 
\subsection*{Reporting of Task 4.3}


\subsection*{Reporting of Task 4.4}


\subsection*{Reporting of Task 4.5}


\subsection*{Reporting of Task 4.6}


\subsection*{Reporting of Task 4.7}

\vdots



\end{document}

